\documentclass[12pt]{article}
\usepackage{algorithmic}

\title{CS 1510: Homework 13}
\author{
John Hofrichter\\\texttt{jmh162@pitt.edu}
\and
Kyra F. Lee \\\texttt{kfl5@pitt.edu}
\and
Zach Sadler \\\texttt{zps6@pitt.edu}}
\date{\today}
\begin{document}
\maketitle
\section*{Dynamic Programming Problem 16}
	\subsection*{Explanation}
	
The first thing to note in this problem is that our bounds for runtime are very large. For instance, if our input is $1, 2, 2, 3, 4, 5, 6, 7$ then we have 8 numbers total ($n = 8$) and L = max($792, 10080) = 10080$. So to solve this small input of simply 8 single-digit numbers we are required to be within polynomial of over $10,000$ operations, which is enormous compared to $2^8 = 256$, which is exponential runtime and typically considered to be way too slow for an acceptable solution.\\
So we understand that even an exponential solution may be a good one compared to our acceptable bounds, so what exactly will we have for our solution? We've already seen that $2^n$ can be a better runtime than the bounds we're given, and in fact our solution will run in about $O(2^n)$.\\
We will calculate the power set of our input and then calculate the cubic sum and total product of each subset to see if it is a solution. If it is, then we finish and return that subset as a valid solution. So consider the following algorithm:\\

	\subsection*{Algorithm}
	
\begin{algorithmic}
	\STATE{PowerSet = null}
	\FOR{i = 1 to n}
		\STATE{newPowerSet = null}
		\FOR{subset 1 to PowerSet.size}
			\STATE{newPowerSet.add(subset)}
			\STATE{newSubset = subset + i}
			\STATE{newPowerSet.add(newSubset)}
		\ENDFOR
		\STATE{PowerSet = newPowerSet}
	\ENDFOR
	\STATE{//So now we have PowerSet as our complete power set of the input.}
	\FOR{i = 1 to PowerSet.size}
		\IF{cubicSum(PowerSet[i]) == totalProduct(PowerSet[i])}
			\STATE{return i}
		\ENDIF
	\ENDFOR
	\STATE{return "no solution"}
\end{algorithmic}	


\end{document}